\documentclass[11pt]{article}
\usepackage[utf8]{inputenc}
\usepackage[T1]{fontenc}
\usepackage{listings}
\usepackage{color}
\usepackage{qtree}
\usepackage{amsmath}
\usepackage{mathtools}

\title{\textbf{MC558 - Análise de Algoritmos}}
\author{João Henrique Dalben}

\definecolor{codegreen}{rgb}{0,0.6,0}
\definecolor{codegray}{rgb}{0.5,0.5,0.5}
\definecolor{codepurple}{rgb}{0.58,0,0.82}
\definecolor{backcolour}{rgb}{0.95,0.95,0.92}

\lstdefinestyle{mystyle}{
    backgroundcolor=\color{backcolour},   
    commentstyle=\color{codegreen},
    keywordstyle=\color{magenta},
    numberstyle=\tiny\color{codegray},
    stringstyle=\color{codepurple},
    basicstyle=\footnotesize,
    breakatwhitespace=false,         
    breaklines=true,                 
    captionpos=b,                    
    keepspaces=true,                 
    numbers=left,                    
    numbersep=5pt,                  
    showspaces=false,                
    showstringspaces=false,
    showtabs=false,                  
    tabsize=2
}

\lstset{style=mystyle}



\begin{document}

\maketitle

\section*{Lista 1}

\subsection*{22.1-1}

Given an adjacency-list representation of a directed graph, how long does it take
to compute the out-degree of every vertex?  How long does it take to compute the
in-degrees?


\begin{lstlisting}
OUT-DEGREE(G)
	for each vertex v in G.V
		v.out_degree = 0
		for each vertex u in G.Adj[v]
			v.out_degree++
\end{lstlisting}

\begin{lstlisting}
IN-DEGREE(G)
	for each vertex v in G.V
		v.in_degree = 0
		for each vertex u in G.Adj[v]
			u.in_degree++
\end{lstlisting}

According to the book, "if $G$ is a directed graph, the sum of the lengths of all the adjancecy lists is $|E|$". Therefore, as both OUT-DEGREE and IN-DEGREE algorithms visit every vertex and every entry of every adjacency list once, their complexity is $\Theta(V+E)$.

\subsection*{22.1-2}
Give an adjacency-list representation for a complete binary tree on 7 vertices. Give an equivalent adjacency-matrix representation. Assume that vertices are numbered from 1 to 7 as in a binary heap.

Tree representation:

\Tree[.1 [.2 [.4 ]
               [.5 ]]
          [.3 [.6 ]
                [.7 ]]]


Adjacency-matrix representation:            
\[
 \begin{matrix}
  & \textbf{1} & \textbf{2} & \textbf{3} & \textbf{4} & \textbf{5} & \textbf{6} & \textbf{7} \\
  \textbf{1} & 0 & 1 & 1 & 0 & 0 & 0 & 0 \\
  \textbf{2} & 0 & 0 & 0 & 1 & 1 & 0 & 0 \\
  \textbf{3} & 0 & 0 & 0 & 0 & 0 & 1 & 1 \\
  \textbf{4} & 0 & 0 & 0 & 0 & 0 & 0 & 0 \\
  \textbf{5} & 0 & 0 & 0 & 0 & 0 & 0 & 0 \\
  \textbf{6} & 0 & 0 & 0 & 0 & 0 & 0 & 0 \\
  \textbf{7} & 0 & 0 & 0 & 0 & 0 & 0 & 0 \\
 \end{matrix}
\]

\subsection*{22.1-3}

The transpose of a directed graph $G = (V, E)$  is the graph $G^T = (V, E^T)$, where $E^T = \{ (v,u) \in V \times V : (u,v) \in E\}$. Thus, $G^T$ is $G$ with all its edges reversed. Describe efficient algorithms for computing $G^T$ from $G$, for both the adjacency-list and adjacency-matrix representations of $G$. Analyze the running times of your algorithms.

\begin{lstlisting}
TRANSPOSE-LIST(G)
	G_t.V = G.V
	for each vertex v in G.V
		for each vertex u in G.Adj[v]
			G_t.Adj[u].append(v)
	return G_t
\end{lstlisting}

\begin{lstlisting}
TRANSPOSE-MATRIX(G)
	G_t = G
	for i from 1 to |G.V|
		for j from i+1 to |G.V| 
			swap(G_t[i][j], G_t[j][i])
\end{lstlisting}

In TRANSPOSE-LIST, the complexity analysis is the same as exercise 22.1-1, which is $O(V+E)$.

In TRANSPOSE-MATRIX, the nested loop between lines 3 and 5 has $\sum_{n=1}^{V} V - i$ (consider V = |V|) iterations, and its complexity is

\begin{equation}
\begin{split}
\sum_{n=1}^{V} V - i = \sum_{n=1}^{V} V - \sum_{n=1}^{V} i = V^2 - \frac{V(V+1)}{2} =\\V^2 - \frac{V^2}{2} - \frac{V}{2} = \frac{V^2}{2} - \frac{V}{2} = \frac{V(V-1)}{2} = \Theta(V^2)
\end{split}
\end{equation}


\subsection*{22.1-4}

Given an adjacency-list  representation  of a multigraph $G = (V, E)$, describe an $O(V + E)$-time  algorithm  to  compute  the  adjacency-list  representation  of  the “equivalent” undirected graph $G ' = (V, E')$, where $E'$ consists of the edges in E with all multiple edges between two vertices replaced by a single edge and with all self-loops removed.

\begin{lstlisting}
EQUIVALENT(G)
	G'.V = G.V
	for each vertex v in G.V
		for each vertex u in G.Adj[v]
			if (u != v) and (u not in G'.Adj[v])
				G'.Adj[v].append(u)
				G'.Adj[u].append(v)
	return G'
\end{lstlisting}

\subsection*{22.1-5}

The square of a directed graph $G = (V, E)$ is the graph $G^2 = (V, E^2)$ such that $(u, v) \in E^2$ if and only $G$ contains a path with at most two edges between $u$ and $v$. Describe efficient algorithms  for computing $G^2$ from $G$ for both the adjacency-list and adjacency-matrix representations of $G$. Analyze the running times of your algorithms.

\begin{lstlisting}
SQUARE-MATRIX(G)
	G' = G
	for i from 1 to |G.V|
		for j from 1 to |G.V|
			G'[i][j] = 0
			for k from 1 to |G.V|
				G'[i][j] = G[i][k]*G[k][j] 
	return G'
\end{lstlisting}

\begin{lstlisting}
SQUARE-LIST(G)
	G'.V = G.V
	for each vertex v in G.V
		for each vertex u in G.Adj[v]
			G'.Adj[v].append(G.Adj[u])
	for each vertex v in G.V
		 COUNTING-SORT(G'.Adj[v])
		 REMOVE-DUPLICATES(G'.Adj[v])
	return G'
\end{lstlisting}
xx'
SQUARE-MATRIX has complexity $O(V^3)$, while SQUARE-LIST has complexity $O(VE)$.


\subsection*{22.1-6}





\end{document}
